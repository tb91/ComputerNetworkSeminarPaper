\documentclass[a4paper,twoside]{IEEEtran}




% Löschen oder kommentieren Sie die folgenden beiden Zeilen aus,
% wenn Sie den Text in Englisch schreiben wollen.
\usepackage{german}
\usepackage[utf8]{inputenc}

\usepackage{ dsfont }
\usepackage{ gensymb }
\usepackage{amsthm}
\usepackage{amsmath}

\newtheorem{boundedSpannerTheorem}{Theorem}[section]




\newcommand{\seminarteilnehmer}{Tim Budweg}
\newcommand{\seminartitel}{Review}

\begin{document}

\title{\seminartitel}
\author{\seminarteilnehmer}

\markboth{Seminar Rechnernetze, Wintersemester 2014/2015}%
{\seminarteilnehmer: \seminartitel}


\maketitle

\begin{abstract}
Diese Review bezieht sich auf das Paper von Carl Rainer Brenk zum Thema \glqq Lokale Spanner-Konstruktion für Ad-hoc-Netzwerke
mit variabler Übertragungsreichweite\grqq
\end{abstract}

\section{Zusammenfassung}
Das Paper handelt von einer Spannerkonstruktion mithilfe eines zentralen Algorithmus und wie man dies zu einem lokalen Algorithmus erweitern kann.

Zu Beginn werden einige Grundlagen und fortgeschrittenes Wissen vermittelt.
Danach wird der Algorithmus anhand eines Beispiels erläutert.
Dazu werden einige Parameter erklärt und kontinuierlich der nächste Schritt erläutert.
Zuerst teilt man anhand der Kantenlänge die Kanten in verschiedene Hierarchie-Ebenen auf.
Der Algorithmus betrachtet alle Ebenen beginnend mit der kleinsten Ebene, welche die kürzesten Kanten beinhaltet. 
Es gibt zwei Möglichkeiten wie mit der aktuellen Kante verfahren wird.

Im Anschluss legt der Autor den Dehnungsfaktor fest und beweist diesen.
Die Beweisidee ist die einer vollständigen Induktion.

Als nächster Teil der Arbeit wird dieser zentralisierte Algorithmus auf einen verteilten, lokalen Algorithmus erweitert.
Dabei wird angenommen, dass jeder Knoten seine 1-Hop-Nachbarschaft kennt und die Knoten synchron Nachrichten austauschen.
Das Prinzip des Algorithmus bleibt gleich. 
Die Bestimmung der Kantenlänge wird weiterhin zentralisiert von den Pivot-Knoten erfüllt.

Zum Schluss wird der Algorithmus auf UDG-Graphen erweitert.




\section{Kritik}

\section{Sonstiges}





\end{document}
