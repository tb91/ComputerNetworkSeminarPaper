\documentclass[a4paper,twoside]{IEEEtran}




% Löschen oder kommentieren Sie die folgenden beiden Zeilen aus,
% wenn Sie den Text in Englisch schreiben wollen.
\usepackage{german}
\usepackage[utf8]{inputenc}

\usepackage{ dsfont }
\usepackage{ gensymb }
\usepackage{amsthm}
\usepackage{amsmath}

\newtheorem{boundedSpannerTheorem}{Theorem}[section]




\newcommand{\seminarteilnehmer}{}
\newcommand{\seminartitel}{Review}

\begin{document}

\title{\seminartitel}
\author{\seminarteilnehmer}

\markboth{Seminar Rechnernetze, Wintersemester 2014/2015}%
{\seminarteilnehmer: \seminartitel}


\maketitle

\begin{abstract}
Diese Review bezieht sich auf das Paper von Carl Rainer Brenk zum Thema \glqq Lokale Spanner-Konstruktion für Ad-hoc-Netzwerke
mit variabler Übertragungsreichweite\grqq
\end{abstract}

\section{Zusammenfassung}
Das Paper handelt von einer Spannerkonstruktion mithilfe eines zentralen Algorithmus und wie man dies zu einem lokalen Algorithmus erweitern kann.

Zu Beginn werden einige Grundlagen und fortgeschrittenes Wissen vermittelt.
Danach wird der Algorithmus anhand eines Beispiels erläutert.
Dazu werden einige Parameter erklärt und kontinuierlich der nächste Schritt erläutert.
Zuerst teilt man anhand der Kantenlänge die Kanten in verschiedene Hierarchie-Ebenen auf.
Der Algorithmus betrachtet alle Ebenen beginnend mit der kleinsten Ebene, welche die kürzesten Kanten beinhaltet. 
Es gibt zwei Möglichkeiten wie mit der aktuellen Kante verfahren wird.

Im Anschluss legt der Autor den Dehnungsfaktor fest und beweist diesen.
Die Beweisidee ist die einer vollständigen Induktion.

Als nächster Teil der Arbeit wird dieser zentralisierte Algorithmus auf einen verteilten, lokalen Algorithmus erweitert.
Dabei wird angenommen, dass jeder Knoten seine 1-Hop-Nachbarschaft kennt und die Knoten synchron Nachrichten austauschen.
Das Prinzip des Algorithmus bleibt gleich. 
Die Bestimmung der Kantenlänge wird weiterhin zentralisiert von den Pivot-Knoten erfüllt.

Zum Schluss wird der Algorithmus auf UDG-Graphen erweitert.




\section{Kritik}

Der Anfang -- Einleitung, Grundlagen, Der zentralisierte Algorithmus -- ist dir hervorragend gelungen.
Du hast gut auf das Thema zugeführt und deinen Gedankengängen kann man gut folgen.
Gegen Ende wird der Text unverständlicher. 
Selbst nach mehrmaligem Lesen habe ich nicht genau verstanden wie die lokale Variante funktioniert.
Meinem Verständnis nach geht das genau wie bei der zentralisierten Version.
Es wird sogar ein \glqq zentralisierte Berechnung der Spanner-Kanten\grqq \space durchgeführt.
Folglich ist die lokale Variante nicht ganz lokal, oder?
Das solltest du erwähnen.
(Wobei man die Menge der ausgehenden Kanten eigentlich lokal bestimmen kann.
Denn laut Annahme kennt jeder seine 1-Hop-Nachbarschaft).

Das Kapitel \glqq Der Algorithmus\grqq \space ist etwas unverständlich. 
Hier sind einige Anmerkungen, die mir dazu eingefallen sind:
\begin{itemize}
\item $u $ und $v $ sind verbunden?
\item Spanner können keine Kanten  \glqq entstehen\grqq \space lassen.
\item Was passiert, wenn die Kante nicht \glqq \space durch bis zu drei Kanten ersetzbar ist?\grqq oder ist das ein Fakt, den du beweisen möchtest?
\end{itemize}

Wie du bereits bemerkt hast, ist der Graph unglücklich gewählt.
Dieses Kapitel ist viel zu lang dafür, dass es immer dasselbe ist. (Fall 2, Fall 2, Fall2,...)

Versuche ein Minimalbeispiel zu finden, welches alle Fälle abdeckt und wenn ein Fall doppelt oder dreifach vorkommt, schreibe sowas wie \glqq analog zu ...\grqq. \space
Am Anfang waren die Abbildungen sehr gut, aber gerade als es zu den Graphen kam, der aufgebaut wird, solltest du noch ein paar Bilder einbauen, welche den aktuellen Zustand des Algorithmus zeigen.


Außerdem hast du sehr viele Füllwörter benutzt, die man beim Sprechen gerne nutzt, aber nicht aufschreiben sollte.
Diese habe ich im Text markiert.

Mir ist aufgefallen, dass der inhaltliche rote Faden deines Papers dem Leser nicht erläutert wird.
Wie wärs, wenn du den Algorithmus einmal in Pseudocode bzw. in kurzen Sätzen beschrieben einbaust.





\section{Sonstiges}
Ich habe einige wenige Rechtschreibfehler gefunden, welche ich angestrichen habe, aber im Großen und Ganzen war der Text fehlerfrei.




\end{document}
